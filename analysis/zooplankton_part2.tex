\documentclass[]{article}
\usepackage{lmodern}
\usepackage{amssymb,amsmath}
\usepackage{ifxetex,ifluatex}
\usepackage{fixltx2e} % provides \textsubscript
\ifnum 0\ifxetex 1\fi\ifluatex 1\fi=0 % if pdftex
  \usepackage[T1]{fontenc}
  \usepackage[utf8]{inputenc}
\else % if luatex or xelatex
  \ifxetex
    \usepackage{mathspec}
  \else
    \usepackage{fontspec}
  \fi
  \defaultfontfeatures{Ligatures=TeX,Scale=MatchLowercase}
\fi
% use upquote if available, for straight quotes in verbatim environments
\IfFileExists{upquote.sty}{\usepackage{upquote}}{}
% use microtype if available
\IfFileExists{microtype.sty}{%
\usepackage{microtype}
\UseMicrotypeSet[protrusion]{basicmath} % disable protrusion for tt fonts
}{}
\usepackage[margin=1in]{geometry}
\usepackage{hyperref}
\hypersetup{unicode=true,
            pdftitle={Zooplancton classes réduites},
            pdfborder={0 0 0},
            breaklinks=true}
\urlstyle{same}  % don't use monospace font for urls
\usepackage{color}
\usepackage{fancyvrb}
\newcommand{\VerbBar}{|}
\newcommand{\VERB}{\Verb[commandchars=\\\{\}]}
\DefineVerbatimEnvironment{Highlighting}{Verbatim}{commandchars=\\\{\}}
% Add ',fontsize=\small' for more characters per line
\usepackage{framed}
\definecolor{shadecolor}{RGB}{248,248,248}
\newenvironment{Shaded}{\begin{snugshade}}{\end{snugshade}}
\newcommand{\KeywordTok}[1]{\textcolor[rgb]{0.13,0.29,0.53}{\textbf{#1}}}
\newcommand{\DataTypeTok}[1]{\textcolor[rgb]{0.13,0.29,0.53}{#1}}
\newcommand{\DecValTok}[1]{\textcolor[rgb]{0.00,0.00,0.81}{#1}}
\newcommand{\BaseNTok}[1]{\textcolor[rgb]{0.00,0.00,0.81}{#1}}
\newcommand{\FloatTok}[1]{\textcolor[rgb]{0.00,0.00,0.81}{#1}}
\newcommand{\ConstantTok}[1]{\textcolor[rgb]{0.00,0.00,0.00}{#1}}
\newcommand{\CharTok}[1]{\textcolor[rgb]{0.31,0.60,0.02}{#1}}
\newcommand{\SpecialCharTok}[1]{\textcolor[rgb]{0.00,0.00,0.00}{#1}}
\newcommand{\StringTok}[1]{\textcolor[rgb]{0.31,0.60,0.02}{#1}}
\newcommand{\VerbatimStringTok}[1]{\textcolor[rgb]{0.31,0.60,0.02}{#1}}
\newcommand{\SpecialStringTok}[1]{\textcolor[rgb]{0.31,0.60,0.02}{#1}}
\newcommand{\ImportTok}[1]{#1}
\newcommand{\CommentTok}[1]{\textcolor[rgb]{0.56,0.35,0.01}{\textit{#1}}}
\newcommand{\DocumentationTok}[1]{\textcolor[rgb]{0.56,0.35,0.01}{\textbf{\textit{#1}}}}
\newcommand{\AnnotationTok}[1]{\textcolor[rgb]{0.56,0.35,0.01}{\textbf{\textit{#1}}}}
\newcommand{\CommentVarTok}[1]{\textcolor[rgb]{0.56,0.35,0.01}{\textbf{\textit{#1}}}}
\newcommand{\OtherTok}[1]{\textcolor[rgb]{0.56,0.35,0.01}{#1}}
\newcommand{\FunctionTok}[1]{\textcolor[rgb]{0.00,0.00,0.00}{#1}}
\newcommand{\VariableTok}[1]{\textcolor[rgb]{0.00,0.00,0.00}{#1}}
\newcommand{\ControlFlowTok}[1]{\textcolor[rgb]{0.13,0.29,0.53}{\textbf{#1}}}
\newcommand{\OperatorTok}[1]{\textcolor[rgb]{0.81,0.36,0.00}{\textbf{#1}}}
\newcommand{\BuiltInTok}[1]{#1}
\newcommand{\ExtensionTok}[1]{#1}
\newcommand{\PreprocessorTok}[1]{\textcolor[rgb]{0.56,0.35,0.01}{\textit{#1}}}
\newcommand{\AttributeTok}[1]{\textcolor[rgb]{0.77,0.63,0.00}{#1}}
\newcommand{\RegionMarkerTok}[1]{#1}
\newcommand{\InformationTok}[1]{\textcolor[rgb]{0.56,0.35,0.01}{\textbf{\textit{#1}}}}
\newcommand{\WarningTok}[1]{\textcolor[rgb]{0.56,0.35,0.01}{\textbf{\textit{#1}}}}
\newcommand{\AlertTok}[1]{\textcolor[rgb]{0.94,0.16,0.16}{#1}}
\newcommand{\ErrorTok}[1]{\textcolor[rgb]{0.64,0.00,0.00}{\textbf{#1}}}
\newcommand{\NormalTok}[1]{#1}
\usepackage{graphicx,grffile}
\makeatletter
\def\maxwidth{\ifdim\Gin@nat@width>\linewidth\linewidth\else\Gin@nat@width\fi}
\def\maxheight{\ifdim\Gin@nat@height>\textheight\textheight\else\Gin@nat@height\fi}
\makeatother
% Scale images if necessary, so that they will not overflow the page
% margins by default, and it is still possible to overwrite the defaults
% using explicit options in \includegraphics[width, height, ...]{}
\setkeys{Gin}{width=\maxwidth,height=\maxheight,keepaspectratio}
\IfFileExists{parskip.sty}{%
\usepackage{parskip}
}{% else
\setlength{\parindent}{0pt}
\setlength{\parskip}{6pt plus 2pt minus 1pt}
}
\setlength{\emergencystretch}{3em}  % prevent overfull lines
\providecommand{\tightlist}{%
  \setlength{\itemsep}{0pt}\setlength{\parskip}{0pt}}
\setcounter{secnumdepth}{5}
% Redefines (sub)paragraphs to behave more like sections
\ifx\paragraph\undefined\else
\let\oldparagraph\paragraph
\renewcommand{\paragraph}[1]{\oldparagraph{#1}\mbox{}}
\fi
\ifx\subparagraph\undefined\else
\let\oldsubparagraph\subparagraph
\renewcommand{\subparagraph}[1]{\oldsubparagraph{#1}\mbox{}}
\fi

%%% Use protect on footnotes to avoid problems with footnotes in titles
\let\rmarkdownfootnote\footnote%
\def\footnote{\protect\rmarkdownfootnote}

%%% Change title format to be more compact
\usepackage{titling}

% Create subtitle command for use in maketitle
\newcommand{\subtitle}[1]{
  \posttitle{
    \begin{center}\large#1\end{center}
    }
}

\setlength{\droptitle}{-2em}

  \title{Zooplancton classes réduites}
    \pretitle{\vspace{\droptitle}\centering\huge}
  \posttitle{\par}
    \author{}
    \preauthor{}\postauthor{}
    \date{}
    \predate{}\postdate{}
  

\begin{document}
\maketitle

{
\setcounter{tocdepth}{2}
\tableofcontents
}
\begin{Shaded}
\begin{Highlighting}[]
\NormalTok{SciViews}\OperatorTok{::}\StringTok{ }\NormalTok{R}
\end{Highlighting}
\end{Shaded}

\begin{verbatim}
## -- Attaching packages ------------------------------------------------------- SciViews::R 1.1.0 --
\end{verbatim}

\begin{verbatim}
## √ SciViews  1.1.0       √ purrr     0.2.4  
## √ chart     1.2.0       √ readr     1.1.1  
## √ flow      1.1.0       √ tidyr     0.8.0  
## √ data.io   1.2.0       √ tibble    1.4.2  
## √ svMisc    1.1.0       √ ggplot2   2.2.1  
## √ forcats   0.3.0       √ tidyverse 1.2.1  
## √ stringr   1.3.0       √ lattice   0.20.35
## √ dplyr     0.7.4       √ MASS      7.3.49
\end{verbatim}

\begin{verbatim}
## -- Conflicts ------------------------------------------------------------ tidyverse_conflicts() --
## x dplyr::filter()                  masks stats::filter()
## x dplyr::lag()                     masks stats::lag()
## x chart::scale_color_continuous()  masks ggplot2::scale_color_continuous()
## x chart::scale_colour_continuous() masks ggplot2::scale_colour_continuous()
## x chart::scale_fill_continuous()   masks ggplot2::scale_fill_continuous()
## x dplyr::select()                  masks MASS::select()
\end{verbatim}

\begin{Shaded}
\begin{Highlighting}[]
\NormalTok{zooplankton <-}\StringTok{ }\KeywordTok{read}\NormalTok{( }\DataTypeTok{file =} \StringTok{"zooplankton"}\NormalTok{, }\DataTypeTok{package =} \StringTok{"data.io"}\NormalTok{, }\DataTypeTok{lang =} \StringTok{"fr"}\NormalTok{)}
\NormalTok{zooplankton_sub <-}\StringTok{ }\KeywordTok{filter}\NormalTok{(zooplankton, class }\OperatorTok\StringTok{ }\KeywordTok{c}\NormalTok{(}\StringTok{"Poecilostomatoid"}\NormalTok{, }\StringTok{"Calanoid"}\NormalTok{, }\StringTok{"Cyclopoid"}\NormalTok{, }\StringTok{"Harpacticoid"}\NormalTok{))}
\end{Highlighting}
\end{Shaded}

\begin{Shaded}
\begin{Highlighting}[]
\KeywordTok{library}\NormalTok{(ggridges)}
\end{Highlighting}
\end{Shaded}

\section{Introduction}\label{introduction}

Cette recherche porte sur l'étude morphologique des zooplanctons de
Tulear et Madagascar, et plus précisément sur les copépodes. Je vais
présenter plusieurs graphiques et test dans ce rapport d'analyses.

\section{Matériel et méthodes}\label{materiel-et-methodes}

Diverses caractéristiques ont été mesurées par analyse d'image avec le
package zooimage et ImageJ sur des échantillons de zooplancton provenant
de Tuléar, à Madagascar.

\section{But}\label{but}

Déterminer l'influence de la sous-classes des copépodes des zooplanktons
de Tuléar dont les ordres sont Poécilostomatoïde, Calanoïdes,
Harpacticoïde et Cyclopoïde, sur leurs paramètres corporels, comme par
exemple leur taille, leur transparence,..

\section{Statistiques}\label{statistiques}

Dans ce rapport je vais utiliser le test de l'ANOVA et le test post-hoc
de Tukey.

ANOVA : ce test consiste émettre l'hypothèse que les moyennes d'un
nombre k de populations (4 dans ce cas ci) sont égales. Lorsqu'on rejète
cette hypothèse car la valeur de P est inférieure au seuil alpha, cela
signifie qu'au moins deux moyennes diffèrent l'une de l'autre mais nous
ne savons pas lesquelles.

Test post-hoc de Tukey : lorsque que l'on rejète H0 après une ANOVA, on
effectue un test post-hoc de Tukey pour savoir quelles moyennes
diffèrent l'une de l'autre. C'est un test de comparaison deux à deux
mais avec un risque de se tromper moindre que dans un test classique de
comparaison pour 2 populations comme le test de t Student par exemple.

\section{Résultats}\label{resultats}

\subsection{Graphique en camembert du nombre d'individus en fonction des
ordres de
copépodes}\label{graphique-en-camembert-du-nombre-dindividus-en-fonction-des-ordres-de-copepodes}

\begin{Shaded}
\begin{Highlighting}[]
\KeywordTok{chart}\NormalTok{(}\DataTypeTok{data =}\NormalTok{ zooplankton_sub, }\OperatorTok{~}\StringTok{ }\KeywordTok{factor}\NormalTok{(}\DecValTok{0}\NormalTok{) }\OperatorTok\StringTok{ }\NormalTok{class) }\OperatorTok{+}
\StringTok{  }\KeywordTok{geom_bar}\NormalTok{(}\DataTypeTok{width =} \DecValTok{1}\NormalTok{) }\OperatorTok{+}\StringTok{ }
\StringTok{  }\KeywordTok{coord_polar}\NormalTok{(}\StringTok{"y"}\NormalTok{, }\DataTypeTok{start =} \DecValTok{0}\NormalTok{) }\OperatorTok{+}
\StringTok{  }\KeywordTok{theme_void}\NormalTok{() }\OperatorTok{+}
\StringTok{  }\KeywordTok{scale_fill_grey}\NormalTok{()}
\end{Highlighting}
\end{Shaded}

\includegraphics{zooplankton_part2_files/figure-latex/unnamed-chunk-4-1.pdf}

On peut voir qu'il y a beaucoup plus de CalanoÏdes étudiés que de
Poécilostomatoïdes, Harpaticoïdes et Cyclopoïdes.

\subsection{Graphique en colonne de la surface de l'individu en fonction
de l'ordre de
copépodes}\label{graphique-en-colonne-de-la-surface-de-lindividu-en-fonction-de-lordre-de-copepodes}

\begin{Shaded}
\begin{Highlighting}[]
\KeywordTok{chart}\NormalTok{(zooplankton_sub, }\DataTypeTok{formula =}\NormalTok{ area }\OperatorTok{~}\StringTok{ }\NormalTok{class) }\OperatorTok{+}
\StringTok{  }\KeywordTok{geom_col}\NormalTok{() }
\end{Highlighting}
\end{Shaded}

\includegraphics{zooplankton_part2_files/figure-latex/unnamed-chunk-5-1.pdf}

On peut observer ici que l'ordre des Calanoïdes a une surface grandement
supérieure aux troix autres ordres.

L'ordre de Poécilostomatoïdes à une surface trois fois inférieure à la
classe des Calanoïdes, mais il a aussi une surface trois fois supérieure
aux ordres des Cyclopoïdes et des Harpacticoïdes qui eux ont
approximativement la même surface.

Cela est surement dû au fait que le nombre d'individus étudiés dans
chaque ordre est grandement différent.

\subsection{Graphique en colonne du périmètre selon l'ordre de
copépodes}\label{graphique-en-colonne-du-perimetre-selon-lordre-de-copepodes}

\begin{Shaded}
\begin{Highlighting}[]
\KeywordTok{chart}\NormalTok{(zooplankton_sub, perimeter }\OperatorTok{~}\StringTok{ }\NormalTok{class) }\OperatorTok{+}
\StringTok{  }\KeywordTok{geom_col}\NormalTok{()}
\end{Highlighting}
\end{Shaded}

\includegraphics{zooplankton_part2_files/figure-latex/unnamed-chunk-6-1.pdf}

En toute logique, on observe pratiquement la même chose que pour le
graphique comparant les surfaces en fonction des ordres. Les Calanoïdes
ont le plus grand périmètre, suivis par les Poécilostomatoïdes qui ont
un périmètre environ 3.6 fois inférieur. Les Cyclopoïdes et
Harpaticoïdes ont pratiquement le même périmètre.

Cela est surement dû au fait que le nombre d'individus étudiés dans
chaque ordre est grandement différent.

\subsection{Boxplot de la taille de l'axe mineur de l'ellipsoïde en
fonction de l'ordre de
copépodes}\label{boxplot-de-la-taille-de-laxe-mineur-de-lellipsoide-en-fonction-de-lordre-de-copepodes}

\begin{Shaded}
\begin{Highlighting}[]
\KeywordTok{chart}\NormalTok{(zooplankton_sub, }\DataTypeTok{formula =}\NormalTok{ minor }\OperatorTok{~}\StringTok{ }\NormalTok{class) }\OperatorTok{+}\StringTok{ }
\StringTok{  }\KeywordTok{geom_boxplot}\NormalTok{() }\OperatorTok{+}
\StringTok{  }\KeywordTok{stat_summary}\NormalTok{(}\DataTypeTok{geom=} \StringTok{"point"}\NormalTok{, }\DataTypeTok{fun.y=} \StringTok{"mean"}\NormalTok{, }\DataTypeTok{color=} \StringTok{"pink"}\NormalTok{, }\DataTypeTok{size=} \DecValTok{2}\NormalTok{) }
\end{Highlighting}
\end{Shaded}

\includegraphics{zooplankton_part2_files/figure-latex/unnamed-chunk-7-1.pdf}

Dans l'ordre des Calanoïdes et des Poécilostomatoïdes, il y a beaucoup
d'invidus présentant une valeur extrême pour l'axe mineur de
l'élipsoïde, ce qui explique pourquoi le point de la moyenne des indidus
ne se situe pas sur la médiane, les valeurs extrêmes faussent la
moyenne.

Par contre, pour les Cyclopoïdes et les Harpaticoïdes, il n'y a pas de
valeurs extrêmes donc la moyenne est très proche de la valeur de la
médiane.

\subsection{Boxplot de la taille de l'axe majeur de l'ellipsoïde en
fonction de l'ordre de
copépodes}\label{boxplot-de-la-taille-de-laxe-majeur-de-lellipsoide-en-fonction-de-lordre-de-copepodes}

\begin{Shaded}
\begin{Highlighting}[]
\KeywordTok{chart}\NormalTok{(zooplankton_sub, }\DataTypeTok{formula =}\NormalTok{ major }\OperatorTok{~}\StringTok{ }\NormalTok{class) }\OperatorTok{+}\StringTok{ }
\StringTok{  }\KeywordTok{geom_boxplot}\NormalTok{() }\OperatorTok{+}
\StringTok{  }\KeywordTok{stat_summary}\NormalTok{(}\DataTypeTok{geom=} \StringTok{"point"}\NormalTok{, }\DataTypeTok{fun.y=} \StringTok{"mean"}\NormalTok{, }\DataTypeTok{color=} \StringTok{"pink"}\NormalTok{, }\DataTypeTok{size=} \DecValTok{2}\NormalTok{) }\OperatorTok{+}
\StringTok{  }\KeywordTok{coord_flip}\NormalTok{() }
\end{Highlighting}
\end{Shaded}

\includegraphics{zooplankton_part2_files/figure-latex/unnamed-chunk-8-1.pdf}

Dans l'ordre des Calanoïdes il y a beaucoup d'invidus présentant une
valeur extrême pour l'axe mineur de l'élipsoïde et dpour les
Poécilostomatoïdes, il n'y a que 2 individus qui présentent des valeurs
extrêmes, mais celles-ci sont vraiment grandes, ce qui explique pourquoi
le point de la moyenne des indidus ne se situe pas sur la médiane et en
est très éloigné, les valeurs extrêmes faussent la moyenne.

Par contre, pour les Cyclopoïdes et les Harpaticoïdes, il n'y a qu'une
seule valeur extrême, donc la moyenne est très proche de la valeur de la
médiane.

\subsection{Graphique de densité de l'élongation en fonction de l'ordre
de
copépodes}\label{graphique-de-densite-de-lelongation-en-fonction-de-lordre-de-copepodes}

\begin{Shaded}
\begin{Highlighting}[]
\KeywordTok{chart}\NormalTok{(zooplankton_sub, }\OperatorTok{~}\StringTok{ }\NormalTok{elongation }\OperatorTok\StringTok{ }\NormalTok{class) }\OperatorTok{+}\StringTok{ }
\StringTok{  }\KeywordTok{geom_density}\NormalTok{(}\DataTypeTok{bins=}\DecValTok{30}\NormalTok{) }
\end{Highlighting}
\end{Shaded}

\begin{verbatim}
## Warning: Ignoring unknown parameters: bins
\end{verbatim}

\includegraphics{zooplankton_part2_files/figure-latex/unnamed-chunk-9-1.pdf}

On peut voir que les Calanoïdes ont une élongation entre 6 et 50 en
majorité avec quelques exceptions au delà de 50.

Les Cyclopoïdes eux ont une élongation entre 12,5 et 45, les
Harpaticoïdes entre 25 et 60.

Les Poécilostomatoïdes eux ont quasiment tous une élongation entre 4 et
18.

\subsection{Graphique en colonne de densité de la densité optique en
fonction de l'ordre de
copépodes}\label{graphique-en-colonne-de-densite-de-la-densite-optique-en-fonction-de-lordre-de-copepodes}

\begin{Shaded}
\begin{Highlighting}[]
\KeywordTok{chart}\NormalTok{(zooplankton_sub,}\DataTypeTok{formula=}\NormalTok{ density }\OperatorTok{~}\StringTok{ }\NormalTok{class  )}\OperatorTok{+}\StringTok{ }
\StringTok{  }\KeywordTok{geom_col}\NormalTok{()}
\end{Highlighting}
\end{Shaded}

\includegraphics{zooplankton_part2_files/figure-latex/unnamed-chunk-10-1.pdf}

Les Calanoïdes ont une densité optique supérieure à 30, ce qui peut
s'expliquer par leur grande taille. Les Poécilostomatoïdes ont une
densité optique presque égale à 15, ils ont une taille moyenne. Les
Cyclopoïdes et les Harpaticoïdes ont une des densités optiques
similaires et inférieures à 5.

\subsection{Graphique en colonne de la circularité selon l'ordre de
copépodes}\label{graphique-en-colonne-de-la-circularite-selon-lordre-de-copepodes}

\begin{Shaded}
\begin{Highlighting}[]
\KeywordTok{chart}\NormalTok{(zooplankton_sub, }\DataTypeTok{formula =}\NormalTok{  circularity }\OperatorTok{~}\StringTok{ }\NormalTok{class ) }\OperatorTok{+}\StringTok{ }
\StringTok{  }\KeywordTok{geom_col}\NormalTok{()}
\end{Highlighting}
\end{Shaded}

\includegraphics{zooplankton_part2_files/figure-latex/unnamed-chunk-11-1.pdf}

On peut voir que les Calanoïdes et Poécilostomatoïdes ont presque la
même circularité et que les Cyclopïdes et les Harpaticoïdes ont presque
la même circularité entre eux aussi.

\subsection{Statiistiques}\label{statiistiques}

\subsubsection{ANOVA de la surface des individus en fonction de l'ordre
de
copépodes}\label{anova-de-la-surface-des-individus-en-fonction-de-lordre-de-copepodes}

\begin{Shaded}
\begin{Highlighting}[]
\KeywordTok{anova}\NormalTok{(anova. <-}\StringTok{ }\KeywordTok{lm}\NormalTok{(}\DataTypeTok{data =}\NormalTok{ zooplankton_sub, area }\OperatorTok{~}\StringTok{ }\NormalTok{class))}
\end{Highlighting}
\end{Shaded}

\begin{verbatim}
## Analysis of Variance Table
## 
## Response: area
##            Df Sum Sq Mean Sq F value    Pr(>F)    
## class       3  6.481 2.16040  20.783 9.585e-13 ***
## Residuals 531 55.199 0.10395                      
## ---
## Signif. codes:  0 '***' 0.001 '**' 0.01 '*' 0.05 '.' 0.1 ' ' 1
\end{verbatim}

H0 : tous les ordres sont significativement identiques pour la surface
(homoscédasticité)

Ici, nous devons rejeter H0 car la valeur de P est grandement inférieure
au seuil alpha qui est de 5\% (ANOVA, F = 20, ddl = 3 \& 531 valeur P
\textless{}\textless{} 10-3).

\subsubsection{Test post-hoc de Tukey pour la surface des individus en
fonction de l'ordre de
copépodes}\label{test-post-hoc-de-tukey-pour-la-surface-des-individus-en-fonction-de-lordre-de-copepodes}

\begin{Shaded}
\begin{Highlighting}[]
\KeywordTok{summary}\NormalTok{(anovaComp. <-}\StringTok{ }\KeywordTok{confint}\NormalTok{(multcomp}\OperatorTok{::}\KeywordTok{glht}\NormalTok{(anova.,}
  \DataTypeTok{linfct =}\NormalTok{ multcomp}\OperatorTok{::}\KeywordTok{mcp}\NormalTok{(}\DataTypeTok{class =} \StringTok{"Tukey"}\NormalTok{))))}
\end{Highlighting}
\end{Shaded}

\begin{verbatim}
## 
##   Simultaneous Tests for General Linear Hypotheses
## 
## Multiple Comparisons of Means: Tukey Contrasts
## 
## 
## Fit: lm(formula = area ~ class, data = zooplankton_sub)
## 
## Linear Hypotheses:
##                                      Estimate Std. Error t value Pr(>|t|)
## Cyclopoid - Calanoid == 0            -0.27794    0.04940  -5.627   <0.001
## Harpacticoid - Calanoid == 0         -0.10660    0.05501  -1.938   0.2020
## Poecilostomatoid - Calanoid == 0     -0.21313    0.03192  -6.677   <0.001
## Harpacticoid - Cyclopoid == 0         0.17134    0.06888   2.488   0.0586
## Poecilostomatoid - Cyclopoid == 0     0.06481    0.05232   1.239   0.5876
## Poecilostomatoid - Harpacticoid == 0 -0.10653    0.05765  -1.848   0.2395
##                                         
## Cyclopoid - Calanoid == 0            ***
## Harpacticoid - Calanoid == 0            
## Poecilostomatoid - Calanoid == 0     ***
## Harpacticoid - Cyclopoid == 0        .  
## Poecilostomatoid - Cyclopoid == 0       
## Poecilostomatoid - Harpacticoid == 0    
## ---
## Signif. codes:  0 '***' 0.001 '**' 0.01 '*' 0.05 '.' 0.1 ' ' 1
## (Adjusted p values reported -- single-step method)
\end{verbatim}

\begin{Shaded}
\begin{Highlighting}[]
\NormalTok{.oma <-}\StringTok{ }\KeywordTok{par}\NormalTok{(}\DataTypeTok{oma =} \KeywordTok{c}\NormalTok{(}\DecValTok{0}\NormalTok{, }\DecValTok{9}\NormalTok{, }\DecValTok{2}\NormalTok{, }\DecValTok{0}\NormalTok{)); }\KeywordTok{plot}\NormalTok{(anovaComp.); }\KeywordTok{par}\NormalTok{(.oma); }\KeywordTok{rm}\NormalTok{(.oma)}
\end{Highlighting}
\end{Shaded}

\includegraphics{zooplankton_part2_files/figure-latex/unnamed-chunk-14-1.pdf}

Suite au rejet de H0 dans le test de l'ANOVA, un test poc-hoc s'annonce.
En effet, si l'on sait qu'au moins deux moyennes diffèrent, nous ne
savons toujours pas lesquelles.

Le test post-hoc de tukey montre ici sur le graphique que les
Cyclopoïdes et Calanoïdes diffèrent significativement entre eux, ainsi
que les Calanoïdes et les Poécilostomatoïdes entre eux, concernant la
surface. Toutes les autres comparaisons deux à deux ne sont pas
significativement différentes.

Pour résumer, les seuls ordres à différer significativement entre eux
sont les Cyclopoïdes et Calanoïdes ainsi que les Calanoïdes et les
Poécilostomatoïdes.

\subsubsection{ANOVA de la densité optique des individus en fonction de
l'ordre de
copépodes}\label{anova-de-la-densite-optique-des-individus-en-fonction-de-lordre-de-copepodes}

\begin{Shaded}
\begin{Highlighting}[]
\KeywordTok{anova}\NormalTok{(anova. <-}\StringTok{ }\KeywordTok{lm}\NormalTok{(}\DataTypeTok{data =}\NormalTok{ zooplankton_sub, density }\OperatorTok{~}\StringTok{ }\NormalTok{class))}
\end{Highlighting}
\end{Shaded}

\begin{verbatim}
## Analysis of Variance Table
## 
## Response: density
##            Df  Sum Sq Mean Sq F value    Pr(>F)    
## class       3 0.37772 0.12591  22.482 1.023e-13 ***
## Residuals 531 2.97381 0.00560                      
## ---
## Signif. codes:  0 '***' 0.001 '**' 0.01 '*' 0.05 '.' 0.1 ' ' 1
\end{verbatim}

H0 : tous les ordres sont significativement identiques pour la densité
optique (homoscédasticité)

Ici, nous devons rejeter H0 car la valeur de P est grandement inférieure
au seuil alpha qui est de 5\% (ANOVA, F = 20, ddl = 3 \& 531 valeur P
\textless{}\textless{} 10-3).

\subsubsection{Test post-hoc de Tukey pour la densité optique des
individus en fonction de l'ordre de
copépodes}\label{test-post-hoc-de-tukey-pour-la-densite-optique-des-individus-en-fonction-de-lordre-de-copepodes}

\begin{Shaded}
\begin{Highlighting}[]
\KeywordTok{summary}\NormalTok{(anovaComp. <-}\StringTok{ }\KeywordTok{confint}\NormalTok{(multcomp}\OperatorTok{::}\KeywordTok{glht}\NormalTok{(anova.,}
  \DataTypeTok{linfct =}\NormalTok{ multcomp}\OperatorTok{::}\KeywordTok{mcp}\NormalTok{(}\DataTypeTok{class =} \StringTok{"Tukey"}\NormalTok{))))}
\end{Highlighting}
\end{Shaded}

\begin{verbatim}
## 
##   Simultaneous Tests for General Linear Hypotheses
## 
## Multiple Comparisons of Means: Tukey Contrasts
## 
## 
## Fit: lm(formula = density ~ class, data = zooplankton_sub)
## 
## Linear Hypotheses:
##                                       Estimate Std. Error t value Pr(>|t|)
## Cyclopoid - Calanoid == 0            -0.087822   0.011465  -7.660  < 0.001
## Harpacticoid - Calanoid == 0         -0.041973   0.012769  -3.287  0.00575
## Poecilostomatoid - Calanoid == 0     -0.029312   0.007409  -3.956  < 0.001
## Harpacticoid - Cyclopoid == 0         0.045849   0.015988   2.868  0.02062
## Poecilostomatoid - Cyclopoid == 0     0.058510   0.012143   4.818  < 0.001
## Poecilostomatoid - Harpacticoid == 0  0.012661   0.013381   0.946  0.76934
##                                         
## Cyclopoid - Calanoid == 0            ***
## Harpacticoid - Calanoid == 0         ** 
## Poecilostomatoid - Calanoid == 0     ***
## Harpacticoid - Cyclopoid == 0        *  
## Poecilostomatoid - Cyclopoid == 0    ***
## Poecilostomatoid - Harpacticoid == 0    
## ---
## Signif. codes:  0 '***' 0.001 '**' 0.01 '*' 0.05 '.' 0.1 ' ' 1
## (Adjusted p values reported -- single-step method)
\end{verbatim}

\begin{Shaded}
\begin{Highlighting}[]
\NormalTok{.oma <-}\StringTok{ }\KeywordTok{par}\NormalTok{(}\DataTypeTok{oma =} \KeywordTok{c}\NormalTok{(}\DecValTok{0}\NormalTok{, }\DecValTok{9}\NormalTok{, }\DecValTok{2}\NormalTok{, }\DecValTok{0}\NormalTok{)); }\KeywordTok{plot}\NormalTok{(anovaComp.); }\KeywordTok{par}\NormalTok{(.oma); }\KeywordTok{rm}\NormalTok{(.oma)}
\end{Highlighting}
\end{Shaded}

\includegraphics{zooplankton_part2_files/figure-latex/unnamed-chunk-17-1.pdf}

Le test post-hoc de tukey montre ici sur le graphique que les
Cyclopoïdes et Calanoïdes diffèrent significativement entre eux, ainsi
que les Calanoïdes et les Poécilostomatoïdes entre eux, mais aussi les
Calanoïdes et les Harpacticoïdes, concernant la densité optique. Toutes
les autres comparaisons deux à deux ne sont pas significativement
différentes.

Pour résumer, les seuls ordres à différer significativement entre eux
sont les Cyclopoïdes et Calanoïdes, les Calanoïdes et les Harpaticoïdes,
ainsi que les Calanoïdes et les Poécilostomatoïdes.

\section{Conclusion}\label{conclusion}

Les graphiques montrent presque tous de ``grosses'' différences entre
les Calanoïdes et les autres ordres de Copépodes. Les test d'ANOVA et de
Tukey confirme ce que disent les graphiques en montrant que les
Calanoïdes diffèrent significativement des Cyclopoïdes et des
Poécilostomatoïdes pour l'aire et significativement différents des
Harpaticoïdes, Cyclopoïdes et des Poécilostomatoïdes.

Je conclue que l'ordre de copépodes influe sur les paramètres corporels
des individus.


\end{document}
